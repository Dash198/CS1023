\documentclass{acmart}
\begin{document}
    \section*{Newton's Laws of Motion}
    Newton defined the three fundamental laws of motion, given below:
    \subsection{Newton's First Law of Motion}
    \textit{Definition of Inertia:}\\
    An object tends to retain its state of motion, i.e., stays in either rest or moves at constant velocity unless acted upon by an external force.\\
    \textit{This property of the object is called inertia.}

    \subsection{Newton's Second Law of Motion}
    The rate of change of momentum of a body is called the \textit{force}, and it is directly proportion to the rate of change of its velocity, i.e., its \textit{acceleration}.\\
    This can be written mathematically as:\\
    \[
        \overrightarrow{p} = m\overrightarrow{v}\\
    \]
    \[
        \frac{\delta\overrightarrow{p}}{\delta t} = m\frac{\delta\overrightarrow{v}}{\delta t}\\
    \]
    \[
        \overrightarrow{F} = m\overrightarrow{a}
    \]


    In simpler terms (Not using vectors)

    \[
        F = \frac{m_1.V_1-m_0.V_0}{t_1-t_0}
    \]   
    \[
        F = \frac{m(V_1-V_0)}{t_1-t_0}
    \]
    \[
        F = \frac{\Delta V}{\Delta t}
    \]  
    \[
        F = ma
    \]

    \subsection{Newton's Third Law of Motion}
    This law states that every action has an \textit{equal and opposite} reaction.
    

\end{document}